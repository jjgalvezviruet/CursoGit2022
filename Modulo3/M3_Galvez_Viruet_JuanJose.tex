%% PREÁMBULO %%
%El texto de latex en español más simple que te vas a encontrar
\documentclass{article} % Tipo de documento, en este caso artículo
\usepackage[spanish]{babel} % Para opciones propias del idioma, como escribir <<>> para comentar 

\usepackage[utf8]{inputenc} % Para que Latex reconozca caracteres como la ñ, en realidad esto no es necesario, la codificación por defecto es la utf8

\begin{document}
\title{Comentarios sobre los paquetes inputenc y babel}
\author{Juan José Gálvez Viruet}
\date{\today}
\maketitle
Para escribir correctamente en español hay que usar los paquetes <<inputenc>> y <<babel>>:
\begin{itemize}
\item[$-$] inputenc: Para la codificación del documento, si la codificación por defecto es ASCII no se reconoceran tildes ni letras como la ñ, en ese caso debe cambiarse a utf8 con <<usepackage[utf8]\{inputenc\}>>
\item[$-$] babel: Para las características propias del idioma como la forma de entrecomillar o la tabulación al final de un párrafo. Por ejemplo, si no introducimos babel no aparecen las comillas angulares anteriores, sino ¡¡inputenc¿¿ y ¡¡babel¿¿.
\end{itemize}
\end{document}