\documentclass[a4paper,11pt]{article}

% Paquete de idioma
\usepackage[spanish]{babel}

% Codificación utf8
\usepackage[utf8]{inputenc}


\begin{document}
\title{Documento de pruebas módulo 3}
\date{\today}
\author{Juan José GV}
\maketitle

Hola, un salto de línea normal, 
no produce cambio alguno, pero sirve para estructurar el texto en el archivo tex
y no hacer líneas muy largas. Para provocar un salto con indentación es necesario dejar una línea en blanco. 

Ahora hemos empezado otro párrafo. Para no introducir indentación se usan dobles barras: \textbackslash\textbackslash \\
y ya tenemos nuestra nueva línea, aunque el resultado puede que no sea del todo satisfactorio.

A continuación queremos hacer una prueba de separación de palabras,~para ello necesitamos una oración larga. Usando el símbolo $\sim$ hemos conseguido que el primer renglón de este párrafo no termine en <<palabras,>>

La estructura del texto cambio según el idioma, en la siguiente estructura el último punto después de los números de secciones y subsecciones no aparece si no se introduce el paquete babel con español.

\part{Primera parte}
Una primera parte, el libro ``An Introduction of QFT" de Michael E. Peskin y Daniel V. Schroeder tiene tres partes
\begin{itemize}
\item[$1$]Diagramas de Feynman y electodinámica cuántica.
\item[$2$]Renormalización
\item[$3$]Teorías gauge no abelianas
\end{itemize}

\section{Primera sección de la primera parte}
Además tiene muchos capítulos, y dentro de cada capítulo otras tantas secciones.
\subsection{Primera subsección}
Es un libro bastante bien estructurado.


\end{document}